Contents
Using the “Apply” family of functions	1
Complex Simulation Program	2
Cumulative Frequency Chart	3
The Birthday function	4

Using the “Apply” family of functions

The "apply" family of functions keep you from having to write loops to perform some operation on every row or every column of a matrix or data frame, or on every element in a list.

The “apply()” function
The apply() function is a powerful device that operates on arrays and, in particular, matrices.
The apply() function returns a vector (or array or list of values) obtained by applying a specified function to either the row or columns of an array or matrix.
To specify use for rows or columns, use the additional argument of 1 for rows, and 2 for columns.

# create a matrix of 10 rows x 2 columns
m <- matrix(c(1:10, 11:20), nrow = 10, ncol = 2)

# mean of the rows

apply(m, 1, mean)
# [1]  6  7  8  9 10 11 12 13 14 15

# mean of the columns
apply(m, 2, mean)
#[1]  5.5 15.5

The local version of apply()is lapply(), which computes a function for each argument of a list., provided each argument is compatible with the function argument (e.g. that is numeric). 

The lapply() command returns a list of the same length as a list “X”, each element of which is the result of applying a specified function to the corresponding element of X.

A user friendly version of lapply() is sapply().

The sapply() command  is a variant of lapply() – returning a matrix instead of a list - again of the same length as a list X, each element of which is the result of applying a specified function to the corresponding element of X.

> x <- list(a=1:10, b=exp(-3:3), logic=c(T,F,F,T))
>
> # compute the list mean for each list element
>
> lapply(x,mean)
$a
[1] 5.5

$b
[1] 4.535125

$logic
[1] 0.5
>
> sapply(x,mean)
       a        b    logic 
5.500000 4.535125 0.500000
>
Complex Simulation Program

1)	Create a function called SumDice().
2)	Apply SumDice() n times to an initial value x
3)	Graph the data

SumDice = function(x){
	Temp1=sum(sample(1:6,x,replace=TRUE))
	return(Temp1)
	}

X=sapply(rep(100,100),SumDice)

DF1=data.frame(Index=1:1000,Sum=X)


Histogram and Cumulative Frequency Chart
Range of values quite interesting!!
Palette = c("blue","yellow","red","green")
hist(X,breaks=60:80*5,col=Palette)

\end{document}

 

 




 


 

