%----------------------------------------------------------------%
\newpage 

\subsection{Nested For Loops}

A nested FOR loops is simply one FOR loop nested inside another, and is quite useful for performing operations on a two dimensional grid or matrix.

Let us specify a $6 \times 6$ dimensional array, called \textbf{\textit{TwoDice}}, which has as its elements the sum of the row and column numbers. (This is used to represent the sample space (i.e. each possible outcome) from rolls of two dice.

\[  \mbox{TwoDice}[i,j] = i+j \]


\begin{framed}
\begin{verbatim}
# 6 sided dice
Die = 6

#prepare the output
TwoDice = matrix(0,nrow=Die,ncol=Die)

# i is the index for rows
 for(i in 1:Die)
   {
   #j is the index for columns
   for(j in 1:Die)
      {
      #simply assign the sum of the row and column to that element.
      TwoDice[i,j] = i+j
      }
   }

#Print out all the possible values
TwoDice
\end{verbatim}
\end{framed}

Different summations have different probabilities. We can compute the \textbf{textit{probability distribution}} using the following code:




\begin{framed}
\begin{verbatim}
table(TwoDice)
sum(table(TwoDice))

prob2Dice = table(TwoDice)/sum(table(TwoDice))

#round the output to 4 decimal places

round(prob2Dice ,4)
\end{verbatim}
\end{framed}

\textbf{Exercise} Compute the \textbf{Expected Value} of this experiment.

\begin{itemize}
\item \texttt{sum()} sum of items in a vector
\item \texttt{dim()} dimensions of a data object.
\item \texttt{prod()} product of items in a vector.
\end{itemize}



\begin{verbatim}
> sum(TwoDice)
[1] 252
> dim(TwoDice)
[1] 6 6
> prod(dim(TwoDice))
[1] 36
> sum(TwoDice)/prod(dim(TwoDice))
[1] 7
\end{verbatim}

\textbf{Exercise} Suppose we are interested in the absolute difference of the values of two 8-sided dice. What is the probability distribution? What is the expected value?

%------------------------------------------------------------------------------%
