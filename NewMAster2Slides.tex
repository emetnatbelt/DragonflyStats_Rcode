 
\documentclass[pdf,default,slideColor,colorBG]{prosper}
\usepackage{natbib}
\usepackage{vmargin}
\usepackage{graphicx}
\usepackage{epsfig}
\usepackage{subfigure}
%\usepackage{amscd}
\usepackage{amssymb}
\usepackage{subfigure}
\usepackage{amsbsy}
\usepackage{amsthm, amsmath}
%\usepackage[dvips]{graphicx}
\bibliographystyle{chicago}
\renewcommand{\baselinestretch}{1.4}
% define a new font called goodfont
\def\goodfont{\usefont{T1}{pcr}{b}{n}\fontsize{36pt}{40pt}\selectfont\green}
\renewcommand{\familydefault}{\rmdefault}
\renewcommand{\rmdefault}{cmr}
\parindent 0pt
\parskip 5pt

\begin{document}
%---------------------------------------------------------------------------------%
\begin{slide}{Introduction to R}
\begin{itemize}
\item data
\end{itemize}
\end{slide}
%---------------------------------------------------------------------------------%
\begin{slide}{Introduction to R}
\begin{itemize}
\item data
\end{itemize}
\begin{equation}
{n \choose k} = \frac{n!}{k! (n-k)!}
\end{equation}
\end{slide}
%---------------------------------------------------------------------------------%
\begin{slide}{Getting Started}
\begin{itemize}
\item $R$ can be started by clicking the $R$ icon on the computers desktop.
\end{itemize}
\end{slide}
%---------------------------------------------------------------------------------%
\begin{slide}{Help}
\begin{itemize}
\item data
\end{itemize}
\end{slide}
%---------------------------------------------------------------------------------%
\begin{slide}{Command line and comments}
\begin{itemize}
\item the command Line interface \item Comments are added to code
using the hash sign $\#$.
\begin{verbatim}
x=2 # x equals two
y=4 # y equals four
x*y # x multiplied by y
\end{verbatim}
\end{itemize}
\end{slide}
%---------------------------------------------------------------------------------%
\begin{slide}{Resident Data Sets}
\begin{itemize}
\item The following are some data sets that come as part of the
basic $R$ installation. We will be using them frequently in the
course.
\begin{itemize}
\item iris \item mtcars \item ChickWeights \item airquality
\end{itemize}
\item For more information on each, use the help() command, e.g
help(iris).
\end{itemize}
\end{slide}
%---------------------------------------------------------------------------------%

\end{document}
