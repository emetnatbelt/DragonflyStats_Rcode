#################################################
Weighted Linear Regression and Polynomial Regression

Compute the average value snd standard deviation for Fluoresence at each level of concentration.

Comment up on the standard deviations. Are the variance values uniform ( roughly the same level) ? Write your answer in the submission sheet.

Fit a linear model for this data. Write out the regression equation in your submission sheet.

Using the following R code. sketch the scatterplot.

Repeat the regression model fitting procedure using each observation of Fluoresence. 

Conc=c(0,10,20,30,40,50)
Conc.M=rep(Conc,5)
Fluo.M=c(4,22,44,60,75,104,
3,20,46,63,81,109,
4,21,45,60,79,107,
5,22,44,63,78,101,
4,21,44,63,77,105)


##################################################
Fluo.Matrix=matrix(
c(4,22,44,60,75,104,
3,20,46,63,81,109,
4,21,45,60,79,107,
5,22,44,63,78,101,
4,21,44,63,77,105),
byrow=T,ncol=6)

apply(Fluo.Matrix,2,mean)

#################################################

Fit1 = lm(Fluo.M ~ Conc.M)
summary(Fit1)

Fluo.Mean = apply(Fluo.Matrix,2,mean)
Fit2 = lm(Fluo.Mean ~ Conc)
summary(Fit2)

#################################################

plot(Conc.M,Fluo.M)
plot(Conc.M,Fluo.M,pch=18,col="red")
abline(coef(Fit2))


Perform a weighting linear regression model using the following code.
#################################################


# Quadratic and Cubic Relationships


Abso= c(0.084, 0.183, 0.326, 0.464, 0.643, 0.707, 0.717, 0.734 ,0.749)
Conc= c(0.123, 0.288, 0.562, 0.921, 1.420, 1.717, 1.921, 2.137 ,2.321)

plot(Conc,Abso,pch=18,col="red")

# Generate powers of independent variable - Conc

Conc.squared = Conc^2
Conc.squared

Conc.cubed = Conc^3
Conc.cubed

Polynomial regression is similar to Multiple Linear Regression - the various independent variables are simply powers of an underlying variable. Polynomual regression is useful for curvilinear relationships between variables.

On your submission sheet, draw a sketch of the scatterplot. Comment on the shape of the scatterplot? Is the relationship linear? Is there curvature present?

y=b0+b1 x 
y=b0+b1 x + b2 x2
y=b0+b1 x + b2 x2+ b3 x3


FitA = lm(Abso~Conc)
FitB = lm(Abso~Conc + Conc.squared)
FitC = lm(Abso~Conc + Conc.squared + Conc.cubed)



Using the summary command, write out the regression equation for each of these three models.
The number of asterisks beside the p-value indicates the level of significance of the estimates. How many asterisks beside each estimate?

Consider a system where there is no input ( independent variable x = 0 )  
We can remove the intercept term from the model by additionally specifying “-1” in the R code, which specifies to fit a model without an intercept. 


For the quadratic and cubic models, the regression equations have the following form.
y dependent variable
x underlying independent variable  
bi regression coefficients 
y=b1 x + b2 x2
y=b1 x + b2 x2+ b3 x3

FitB.1 = lm(Abso~Conc + Conc.squared-1)
FitC.1 = lm(Abso~Conc + Conc.squared + Conc.cubed-1)










