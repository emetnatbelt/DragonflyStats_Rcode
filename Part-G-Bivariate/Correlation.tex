Part 2 Strategies for Data project
\begin{itemize}
\item Exploratory Data Analysis

The first part of your report should contain some descriptive statistics and summary values. Also include some tests for normality.

\item{Regression}
You should have a data set with multiple columns, suitable for regression analysis.
Familiarize yourself with the data, and decide which variable is the dependent variable.

Also determine the independent variables that you will use as part of your analysis.

\item{Correlation Analysis}
Compute the Pearson correlation for the dependent variable with the respective independent variables.  As part of your report, mention the confidence interval for the correlation estimate
Choose the independent variables with the highest correlation as your candidate variables.
For these independent variables, perform a series of simple linear regression procedures.
\begin{verbatim}
lm(y~x1)
lm(y~x2)
\end{verbatim}
Comment on the slope and intercept estimates and their respective p-values. Also comment on the coefficient of determination (multiple R squared). Remember to write the regression equations.
Perform a series of multiple linear regressions, using pairs of candidate independent variables.
\begin{verbatim}
lm(y~x1 +x2)
lm(y~x2 +x3)
\end{verbatim}
Again, comment on the slope and intercept estimates, and their respective p-values.
In this instance, compare each of the models using the coefficient of determinations. Which model explains the data best?
\subsection{Analysis of residuals}
Perform an analysis of regression residuals ( you can pick the best regression model from last section).
Are the residuals normally distributed?
	Histogram /  Boxplot / QQ plot / Shapiro Wilk Test
Also you can plot the residuals to check that there is constant variance.
\begin{verbatim}
y=rnorm(10)
x=rnorm(10)
fit1=lm(y~x)
res.fit1 = resid(fit1)
plot(res.fit1)
\end{verbatim}

