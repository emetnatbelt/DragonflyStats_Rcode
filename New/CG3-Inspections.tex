3 Inspecting a Data Set
 dim()
 nrow() and ncol()
 names()
 summary()
 tail()
 head()
 describe() (from the psych package)
3.1 Dimensions of a data set
We have remarked that some data sets are very large. This is perhaps a good place to consider
summary information about data objects. For a simple vector, a useful command to determine
the length (remark: sample size) is the function length().
> Y=4:18
> length(Y)
[1] 15
For more complex data sets ( and data frames which we will see later) , we have several
tools for assessing the size of data.
> dim(iris) # dimensions of data set
[1] 150 5
> nrow(iris) # number of rows
[1] 150
> ncol(iris) # number of columns
[1] 5
11
Coding Grace A Taste of R 8th June 2013
We can also determine the row names and column names using the functions rownames()
and colnames(). If there are no specic row or column names, the command will just return
the indices.
> colnames(iris)
[1] "Sepal.Length" "Sepal.Width" "Petal.Length" "Petal.Width" "Species"
3.2 The summary() command
The command summary() is one of the most useful commands in R. It is a generic function used
to produce result summaries of the results of various functions. The function invokes particular
methods which depend on the class of the rst argument. In other words, R picks out the most
suitable type of summary for that data.
> summary(iris)
Sepal.Length Sepal.Width Petal.Length Petal.Width Species
Min. :4.300 Min. :2.000 Min. :1.000 Min. :0.100 setosa :50
1st Qu.:5.100 1st Qu.:2.800 1st Qu.:1.600 1st Qu.:0.300 versicolor:50
Median :5.800 Median :3.000 Median :4.350 Median :1.300 virginica :50
Mean :5.843 Mean :3.057 Mean :3.758 Mean :1.199
3rd Qu.:6.400 3rd Qu.:3.300 3rd Qu.:5.100 3rd Qu.:1.800
Max. :7.900 Max. :4.400 Max. :6.900 Max. :2.500
>
Summary is particularly useful for manipulating data from more complex data objects.
3.3 Structure of a Data Object
The structure, class and storage mode of an object can be determined using the following
commands. Try out a few.
 str()
 class()
 mode()
> class(Nile)
[1] "ts"
> mode(Nile)
[1] "numeric"
>
> a
12
Coding Grace A Taste of R 8th June 2013
[1] 6
> mode(a)
[1] "numeric"
> class(a)
[1] "numeric"
> str(a)
num 6
>
> class(iris)
[1] "data.frame"
> mode(iris)
[1] "list"
