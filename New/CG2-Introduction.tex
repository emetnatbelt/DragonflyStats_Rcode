2 Introduction to R (Continued)
2.1 Three particularly useful commands
1. help()
2. summary()
3. help.start()
2.2 Changing GUI options
We can change the GUI options using the GUI preferences option on the Edit menu. (Im-
portant when teaching R) A demonstration will be done in class.
2.3 Colours
R supported a massive number of colours. Type in colours() (or colors()) to see what colours
are supported.
2.4 Use of the Semi-Colon Operator
The semi-colon operator at the end of each line of code is not necessary in general, but using it
overcomes errors due to copying and pasting from document soft copies. In other programming
languages, such as Octave, using the semicolon in this way has a distinct purpose.
2.5 The apropos() Function
This function is very useful for determining what functions are available for a particular topic,
although the process requires a great deal of trial and error. Suppose we are looking for a
command to compute the correlation coecient. We would use a very short string (e.g. cor)
that would plausibly be part of useful function names.
apropos("cor")
2.6 History
The command history() is used to obtain the last 25 commands used by R
history()
9
Coding Grace A Taste of R 8th June 2013
2.7 The sessionInfo() Function
To get a description of the version of R and its attached packages used in the current session,
we can use the sessionInfo() function
sessionInfo()
2.8 Time and date functions
The commands Sys.time() and Sys.Date() returns the system's idea of the current date
with and without time. We can perform some simple algebraic calculations to compute time
dierences (i.e. to nd out how long some code took to compile.
> X1=Sys.time()
> #Wait a few seconds
>
> X2=Sys.time()
> X2-X1 Time difference of 8.439614 secs
>
> Sys.Date() [1] "2012-09-01"
2.9 Logical States
Logical states TRUE and FALSE are simply specied as such, all in capital letters. The
shortcuts T and F are also recognized.
2.10 Missing Data
In some cases the entire contents of a vector may not be known. For example, missing data
from a particular data set. A place can be reserved for this by assigning it the special value
NA.
NA is a logical constant of length 1 which contains a missing value indicator. NA stands
for Not Available.
2.11 Files in the Working Directory
It is possibel to call an R script from the working directory, using the source() function.
source("myfunctions.r")
source("mydata.r")
We can also send code put directly to a le in the working directory, using the sink()
command. The rst time the command is used, the name of the created le is specied.
Subsequent commands print output directly to this le, until the command is used again to
cease the operation.
10
Coding Grace A Taste of R 8th June 2013
> sink("IrisSum.txt")
> summary(iris)
> sink()
>
1.16 Listing all items in a workspace
To list all items in an R environment, we use the ls() function. This provides a list of all data
objects accessible. Another useful command is objects().
> ls()
[1] "a" "A" "authors" "b" "books"
[6] "C" "D" "ex1" "Gerb" "Lst"
[11] "m" "m1" "op" "presidents" "r"
[16] "showSmooth" "sm" "sm.3RS" "sm2" "sm3"
[21] "Trig" "Vec1" "x" "X" "x.at"
[26] "x1" "x2" "x3R" "y" "Y"
[31] "y.at"
1.17 Removing items
Sometimes it is desirable to save a subset of your workspace instead of the entire workspace.
One option is to use the rm() function to remove unwanted objects right before exiting your R
session; another possibility is to use the save() function.
1.18 Saving and Loading R Data Objects
In situations where a good deal of processing must be used on a raw dataset in order to prepare
it for analysis, it may be prudent to save the R objects you create in their internal binary form.
One attractive feature of this scheme is that the objects created can be read by R programs
running on dierent computer architectures than the one on which they were created, making it
very easy to move your data between dierent computers. Each time an R session is completed,
you are prompted to save the workspace image, which is a binary le called .RData in the
working directory.
Whenever R encounters such a le in the working directory at the beginning of a session,
it automatically loads it making all your saved objects available again. So one method for
8
Coding Grace A Taste of R 8th June 2013
saving your work is to always save your workspace image at the end of an R session. If you
would like to save your workspace image at some other time during your R session, you can use
the save.image() function, which, when called with no arguments, will also save the current
workspace to a le called .RData in the working directory.

