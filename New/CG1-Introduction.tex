Coding Grace A Taste of R 8th June 2013
Contents
1 Introduction to R 3
1.1 Installing R . . . . . . . . . . . . . . . . . . . . . . . . . . . . . . . . . . . . . . 3
1.2 Command Line Interface . . . . . . . . . . . . . . . . . . . . . . . . . . . . . . . 3
1.3 The Assignment operator . . . . . . . . . . . . . . . . . . . . . . . . . . . . . . . 4
1.3.1 Reserved Words . . . . . . . . . . . . . . . . . . . . . . . . . . . . . . . . 4
1.4 Commenting . . . . . . . . . . . . . . . . . . . . . . . . . . . . . . . . . . . . . . 4
1.5 Dening Variables . . . . . . . . . . . . . . . . . . . . . . . . . . . . . . . . . . . 5
1.6 Help Functions . . . . . . . . . . . . . . . . . . . . . . . . . . . . . . . . . . . . 5
1.7 The help.start() command . . . . . . . . . . . . . . . . . . . . . . . . . . . . 5
1.8 Basic Maths Operations . . . . . . . . . . . . . . . . . . . . . . . . . . . . . . . 5
1.9 Basic R Editor . . . . . . . . . . . . . . . . . . . . . . . . . . . . . . . . . . . . 6
1.10 Built-In Data Sets . . . . . . . . . . . . . . . . . . . . . . . . . . . . . . . . . . 6
1.11 The summary() command . . . . . . . . . . . . . . . . . . . . . . . . . . . . . . 7
1.12 Working directories . . . . . . . . . . . . . . . . . . . . . . . . . . . . . . . . . . 7
1.13 Coming Unstuck . . . . . . . . . . . . . . . . . . . . . . . . . . . . . . . . . . . 7
1.14 Quitting the R environment . . . . . . . . . . . . . . . . . . . . . . . . . . . . . 8
1.15 Data Objects . . . . . . . . . . . . . . . . . . . . . . . . . . . . . . . . . . . . . 8
1.16 Listing all items in a workspace . . . . . . . . . . . . . . . . . . . . . . . . . . . 8
1.17 Removing items . . . . . . . . . . . . . . . . . . . . . . . . . . . . . . . . . . . . 8
1.18 Saving and Loading R Data Objects . . . . . . . . . . . . . . . . . . . . . . . . 8
2 Introduction to R (Continued) 9
2.1 Three particularly useful commands . . . . . . . . . . . . . . . . . . . . . . . . . 9
2.2 Changing GUI options . . . . . . . . . . . . . . . . . . . . . . . . . . . . . . . . 9
2.3 Colours . . . . . . . . . . . . . . . . . . . . . . . . . . . . . . . . . . . . . . . . 9
2.4 Use of the Semi-Colon Operator . . . . . . . . . . . . . . . . . . . . . . . . . . . 9
2.5 The apropos() Function . . . . . . . . . . . . . . . . . . . . . . . . . . . . . . . 9
2.6 History . . . . . . . . . . . . . . . . . . . . . . . . . . . . . . . . . . . . . . . . . 9
2.7 The sessionInfo() Function . . . . . . . . . . . . . . . . . . . . . . . . . . . . 10
2.8 Time and date functions . . . . . . . . . . . . . . . . . . . . . . . . . . . . . . . 10
2.9 Logical States . . . . . . . . . . . . . . . . . . . . . . . . . . . . . . . . . . . . . 10
2.10 Missing Data . . . . . . . . . . . . . . . . . . . . . . . . . . . . . . . . . . . . . 10
2.11 Files in the Working Directory . . . . . . . . . . . . . . . . . . . . . . . . . . . . 10
3 Inspecting a Data Set 11
3.1 Dimensions of a data set . . . . . . . . . . . . . . . . . . . . . . . . . . . . . . . 11
3.2 The summary() command . . . . . . . . . . . . . . . . . . . . . . . . . . . . . . 12
3.3 Structure of a Data Object . . . . . . . . . . . . . . . . . . . . . . . . . . . . . . 12
4 Packages 13
4.1 Packages . . . . . . . . . . . . . . . . . . . . . . . . . . . . . . . . . . . . . . . . 13
4.2 Using and Installing packages . . . . . . . . . . . . . . . . . . . . . . . . . . . . 13
4.2.1 Version of R . . . . . . . . . . . . . . . . . . . . . . . . . . . . . . . . . . 13
1
Coding Grace A Taste of R 8th June 2013
5 Data Creation, Data Entry, Data Import and Export 14
5.1 The c() command . . . . . . . . . . . . . . . . . . . . . . . . . . . . . . . . . . 14
5.1.1 Vector of Numeric Values . . . . . . . . . . . . . . . . . . . . . . . . . . 14
5.1.2 Vector of Character Values . . . . . . . . . . . . . . . . . . . . . . . . . . 14
5.1.3 Vector of Logical Values . . . . . . . . . . . . . . . . . . . . . . . . . . . 14
5.2 The scan() command . . . . . . . . . . . . . . . . . . . . . . . . . . . . . . . . 14
5.2.1 Characters with the scan() command . . . . . . . . . . . . . . . . . . . 15
5.3 Using the data editor . . . . . . . . . . . . . . . . . . . . . . . . . . . . . . . . . 15
5.4 Spreadsheet Interface . . . . . . . . . . . . . . . . . . . . . . . . . . . . . . . . . 15
2
Coding Grace A Taste of R 8th June 2013
1 Introduction to R
Source: R project website (http://www.r-project.org)
R is a language and environment for statistical computing and graphics. It is a GNU project
which is similar to the S language and environment which was developed at Bell Laboratories
(formerly AT&T, now Lucent Technologies) by John Chambers and colleagues. R can be con-
sidered as a dierent implementation of S. There are some important dierences, but much
code written for S runs unaltered under R.
R provides a wide variety of statistical (linear and nonlinear modelling, classical statistical tests,
time-series analysis, classication, clustering, ...) and graphical techniques, and is highly ex-
tensible. The S language is often the vehicle of choice for research in statistical methodology,
and R provides an Open Source route to participation in that activity.
One of R's strengths is the ease with which well-designed publication-quality plots can be
produced, including mathematical symbols and formulae where needed. Great care has been
taken over the defaults for the minor design choices in graphics, but the user retains full control.
R is available as Free Software under the terms of the Free Software Foundation's GNU General
Public License in source code form. It compiles and runs on a wide variety of UNIX platforms
and similar systems (including FreeBSD and Linux), Windows and MacOS.
R is a programming environment
 uses a well-developed but simple programming language
 allows for rapid development of new tools according to user demand
 these tools are distributed as packages, which any user can download to customize the R
environment.
Base R and most R packages are available for download from the Comprehensive R Archive Net-
work (CRAN) cran.r-project.org. Base R comes with a number of basic data management,
analysis, and graphical tools R's power and 
exibility, however, lie in its array of packages
(currently more than 4,000!)
1.1 Installing R
R is very easily installed by downloading it from the CRAN website. Installation usually takes
about 2 minutes. When Installation of R is complete, the distinctive R Icon will appear on your
desktop. To start R, simply click this Icon. It is common to re-install R once a year or so. The
current version of R, version 3.0.0. was released quite recently.
1.2 Command Line Interface
When you start R, the command line interface window will appear on screen. This is one
of many windows in the R environment, others including graphical output windows, or script
3
Coding Grace A Taste of R 8th June 2013
editors. R code can be entered into the command line directly. Alternatively code can be saved
to a script, which can be run inside a session using the source() function.
1.3 The Assignment operator
The assignment operator is used to assign names to particular values. Historically the assign-
ment operator was ) a "<-". The assignment operator can also be =. This is valid as of R
version 1.4.0.
Both will be used, although, you should learn one and stick with it. Many long term R
users prefer the arrow approach. You can also use -> as an assignment operator, reversing the
usual assignment positions. (This is actually really useful). Commands are separated either by
a semi colon or by a newline.
> a <- 6
> b = 5
> a + b ->c
> c
[1] 11
>e=7;f<-4
Before we continue, try using the up and down keys, and see what happens. Previously
typed commands are re-presented, and can be re-executed.
R stores both data and output from data analysis (as well as everything else) in objects.
The variables we have created so far are objects. A list of all objects in the current session can
be obtained with ls().
1.3.1 Reserved Words
Some names are used by the system, e.g.T, F,q,c etc . Avoid using these.
1.4 Commenting
For the sake of readability, it is good practice to comment code. The # character at the
beginning of a line signies a comment, which is not executed. Lines of hashtags can be used
to identify the beginning and end of code segments
# This is a comment
#####################
# Start of Segment 1
#####################
4
Coding Grace A Taste of R 8th June 2013
1.5 Dening Variables
A convention is to use dene a variable name with a capital letter (R is case sensitive). This
reduces the chance of overwriting in-build R functions, which are usually written in lowercase
letters. Commonly used variable names such as x,y and z (in lower case letters) are not \re-
served".
1.6 Help Functions
Help les for R functions are accessed by preceding the name of the function with ? (e.g. ?sort
). Alternatively you can use the command help() (e.g. ?sqrt
A HTML document appears on screen with information on the function typed in. As well
as the list of arguments that the particular function accepts, and how to specify them, there is
example code at the bottom of the le. These code segments are often invaluable in learning
how to master the function.
1.7 The help.start() command
As mentioned by the text on the main console, the help.start() command can be usd to
access detailed help documentation that comes as part of the installation.
1.8 Basic Maths Operations
The most commonly used mathematical operators are all supported by R. Here are a few
examples:
5 + 3 * 5 # Note the order of operations.
log (10) # Natural logarithm with base e=2.718282
log(8,2) # Log to the base 2
4^2 # 4 raised to the second power
7/2 # Division
factorial(4) #Factorial of Four
sqrt (25) # Square root
abs (3-7) # Absolute value of 3-7
pi # The mysterious number \\\
exp(2) # exponential function
R can be used for many mathematical operations, including
 Set Theory
 Trigonometry
 Complex Numbers
 Binomial Coecients
We will not go into any of these in great detail today.
5
Coding Grace A Taste of R 8th June 2013
1.9 Basic R Editor
R has an inbuilt script editor. We will use it for this class, but there are plenty of top quality
Integrated Development Environments out there. (Read up about RStudio for example).
For these workshops, we will use the in-built script editor.
To start a new script, or open an existing script simply go to File and click the appropriate
options. A new dialogue box will appear. You can write and edit code using this editor.
To pass the code for compiling , press the run line or selection option (The third icon
on the menu).
Another way to read code is to use the edit() function , which operates directly from the
command line. To see how the code dening an object X was written (or more importantly,
could have been written) simply type edit(X). This command has some useful applications
that we will see later on.
Scripts are saved as .R les. These scripts can be called directly using the source() com-
mand.
1.10 Built-In Data Sets
Several data sets , intended as learning tools, are automatically installed when R is installed.
Many more are installed within packages to complement learning to use those packages. One
of these is the famous Iris data set, which is used in many data mining exercises.
 iris
 mtcars
 Nile
To see what data sets are available, simply type data(). To load a data set, simply type in the
name of the data set. Some data sets are very large. To just see the rst few (or last) rows, we
use the head() function or alternatively the tail() function. The default number of rows of
these commands is 6. Other numbers can be specied.
> head(iris)
Sepal.Length Sepal.Width Petal.Length Petal.Width Species
1 5.1 3.5 1.4 0.2 setosa
2 4.9 3.0 1.4 0.2 setosa
3 4.7 3.2 1.3 0.2 setosa
4 4.6 3.1 1.5 0.2 setosa
5 5.0 3.6 1.4 0.2 setosa
6 5.4 3.9 1.7 0.4 setosa
>
> tail(iris,4)
Sepal.Length Sepal.Width Petal.Length Petal.Width Species
147 6.3 2.5 5.0 1.9 virginica
148 6.5 3.0 5.2 2.0 virginica
149 6.2 3.4 5.4 2.3 virginica
150 5.9 3.0 5.1 1.8 virginic
6
Coding Grace A Taste of R 8th June 2013
In many situations, it is useful to call a particular data set using the attach() command. This
will save having to specify the data sets over repeated operations. The le can then be detached
using the detach() command.
1.11 The summary() command
The R command summary() is a generic function used to produce result \summaries" of the
results of various objects, from simple vectors to the output of complex model tting functions.
The function invokes particular methods which depend on the class of the rst argument.
> summary(Nile)
Min. 1st Qu. Median Mean 3rd Qu. Max.
456.0 798.5 893.5 919.4 1032.0 1370.0
>
> summary(Indometh)
Subject time conc
1:11 Min. :0.250 Min. :0.0500
4:11 1st Qu.:0.750 1st Qu.:0.1100
2:11 Median :2.000 Median :0.3400
5:11 Mean :2.886 Mean :0.5918
6:11 3rd Qu.:5.000 3rd Qu.:0.8325
3:11 Max. :8.000 Max. :2.7200
1.12 Working directories
You can change your working directly by using the appropriate options on the File menu. To
determine the current working directory, you can use the getwd() command. To change the
working directory , we would use the setwd() command. This is quite important as objects
will be imported and exported to and from the specied directory.
> getwd()
[1] "C:/Users/Kevin"
>
> setwd("C:/Users/Kevin/Documents")
>
> getwd()
[1] "C:/Users/Kevin/Documents"
1.13 Coming Unstuck
If you are having trouble with a piece of code that is currently compiling , all you have to do
is press ESC, just like many other computing environments.
7
Coding Grace A Taste of R 8th June 2013
1.14 Quitting the R environment
As the front page text indicates, all you have to do to quite the workspace is to type in q().
You will then be prompted to save your work.
1.15 Data Objects
As mentioned previously, R saves data as objects. Examples of data objects are
 Vectors
 Lists
 Dataframes
 Matrices
The simple objects we have created previously are simply single element vectors.
