
%==============================================================================%
5 Data Creation, Data Entry, Data Import and Export
5.1 The c() command
To create a simple data set, known as a vector, we use the c() command to create the vector.
> Y=c(1,2,4,8,16 ) #create a data vector with specified elements
> Y
[1] 1 2 4 8 16

%==============================================================================%
5.1.1 Vector of Numeric Values
Numvec = c(10,13,15,19,25);
5.1.2 Vector of Character Values
Charvec = c("LouLou","Oscar","Rasher");


%==============================================================================%
5.1.3 Vector of Logical Values
Charvec = c(TRUE,TRUE,FALSE,TRUE);
Vectors can be bound together either by row or by column.
> X=1:3; Y=4:6
> cbind(X,Y)
X Y
[1,] 1 4
[2,] 2 5
[3,] 3 6
>
> rbind(X,Y)
[,1] [,2] [,3]
X 1 2 3
Y 4 5 6

%==============================================================================%
5.2 The scan() command
The scan() function is a useful method of inputting data quickly. You can use to quickly copy
and paste values into the R environment. It is best used in the manner as described in the
following example. Create a variable "X" and use the scan() function to populate it with
values. Type in a value, and then press return. Once you have entered all the values, press
return again to return to normal operation.
> X=scan()
1: 4
2: 5
3: 5
4: 6
5:
Read 4 items
Remark: Try the edit() command on object X.
%==========================================================================%
5.2.1 Characters with the scan() command
The scan() command can also be used forinputting character data.
> Y=scan(what=" ")
1: LouLou
2: Oscar
3: Charlie
4:
Read 3 items
> Y
[1] "LouLou" "Oscar" "Charlie"

%==============================================================================%
5.3 Using the data editor
%==============================================================================%
5.4 Spreadsheet Interface
R provides a spreadsheet interface for editing the values of existing data sets. We use the
command data.entry(), and name of the data object as the argument. (Also try out the
fix() command)
> data.entry(X) # Edit the data set and exit interface
> X

%==============================================================================%
\end{document}
