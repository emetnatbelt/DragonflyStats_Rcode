\section{ Packages }
%--------------------------------------------------- %
4.1 Packages
%--------------------------------------------------- %
A Package in \texttt{R} is a le containing a collection of objects which have some common purpose.
Packages enhance the capabilties and scope for research in a certain eld. For example the
package MASS contains objects associated with the Venables and Ripleys "\textbf{Modern Applied
Statistics with S}". Some examples of packages are Actuar, written for actuarial science, and
QRMlib, which complements the Quantitative Risk Management.

The command library()
lists all the available packages. To load a particular package, for example MASS, we would
write
library(MASS)
%--------------------------------------------------- %
\subsection{ Using and Installing Packages }
Many packages come with R. To use them in an R session, you need to load the package, as
previously demonstrated.
%--------------------------------------------------- %
Some packages are not automatically installed when you install R but they need to be down-
loaded and installed individually. We must rst install them using the install.packages()
function, which typically downloads the package from CRAN and installs it for use. (follow the
instructions as indicated).
install.packages("ggplot2")
install.packages("qcc")
install.packages("sqldf")
install.packages("RMongo")
install.packages("randomforest")
\subsection{ 4.2.1 Version of R }
Many packages will require you to have the most recent version of R and also other packages.
It is a good idea to update regularly.
 \end{document}