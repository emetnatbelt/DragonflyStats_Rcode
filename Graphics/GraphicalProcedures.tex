% !TEX TS-program = pdflatex
% !TEX encoding = UTF-8 Unicode

% This is a simple template for a LaTeX document using the "article" class.
% See "book", "report", "letter" for other types of document.

\documentclass[11pt]{article} % use larger type; default would be 10pt

\usepackage[utf8]{inputenc} % set input encoding (not needed with XeLaTeX)

%%% Examples of Article customizations
% These packages are optional, depending whether you want the features they provide.
% See the LaTeX Companion or other references for full information.

%%% PAGE DIMENSIONS
\usepackage{geometry} % to change the page dimensions
\geometry{a4paper}
\usepackage{graphicx} % support the \includegraphics command and options

\usepackage{booktabs} % for much better looking tables
\usepackage{array} % for better arrays (eg matrices) in maths
\usepackage{paralist} % very flexible & customisable lists (eg. enumerate/itemize, etc.)
\usepackage{verbatim} % adds environment for commenting out blocks of text & for better verbatim
\usepackage{subfig} 
\usepackage{fancyhdr} % This should be set AFTER setting up the page geometry
\pagestyle{fancy} % options: empty , plain , fancy
\renewcommand{\headrulewidth}{0pt} % customise the layout...
\lhead{}\chead{}\rhead{}
\lfoot{}\cfoot{\thepage}\rfoot{}

%%% SECTION TITLE APPEARANCE
\usepackage{sectsty}
\usepackage{framed}
\allsectionsfont{\sffamily\mdseries\upshape} % (See the fntguide.pdf for font help)
% (This matches ConTeXt defaults)

%%% ToC (table of contents) APPEARANCE
\usepackage[nottoc,notlof,notlot]{tocbibind} % Put the bibliography in the ToC
\usepackage[titles,subfigure]{tocloft} % Alter the style of the Table of Contents
\renewcommand{\cftsecfont}{\rmfamily\mdseries\upshape}
\renewcommand{\cftsecpagefont}{\rmfamily\mdseries\upshape} % No bold!
%---------------------------------------------------%
\begin{document}
\maketitle

\section{Assessing the Distrubution of a Variable}


\subsection{Simple Graphical Tecnhiques}

\begin{itemize}
\item Histograms
\item Density Plots
\item Boxplots
\end{itemize}


%------------------------------------------------------------------------------------------------%
\section{Scatterplots}
\begin{figure}
  % Requires \usepackage{graphicx}
  \includegraphics[scale=0.40]{MTCARSmpgwt.png}\\
  \caption{Scatterplot}\label{mpgwt}
\end{figure}


\section{Adding titles, lines, points to plots}


\footnotesize \begin{verbatim}
library(MASS)
# Colour points and choose plotting symbols according to a levels of a factor
plot(Cars93$Weight, Cars93$EngineSize, col=as.numeric(Cars93$Type),
pch=as.numeric(Cars93$Type))

# Adds x and y axes labels and a title.
plot(Cars93$Weight, Cars93$EngineSize, ylab="Engine Size",
xlab="Weight", main="My plot")
# Add lines to the plot.
lines(x=c(min(Cars93$Weight), max(Cars93$Weight)), y=c(min(Cars93$EngineSize),
max(Cars93$EngineSize)), lwd=4, lty=3, col="green")
abline(h=3, lty=2)
abline(v=1999, lty=4)
# Add points to the plot.
\end{verbatim}\normalsize



%---------------------------------------------------%
\section{Basic Graphical Procedures}







