


R Workshop

Writing Basic Functions

 

Writing Basic Functions

A simple function can be constructed as follows:

 







function_name <- function(arg1, arg2, ...){

commands

output

}

 
 




You decide on the name of the function. The function command shows R that you are writing a function. 

Inside the parenthesis you outline the input objects required and decide what to call them. 

The commands occur inside the { }.

The name of whatever output you want goes at the end of the function. 

Comments lines (usually a description of what the function does is placed at the beginning) are denoted by #.

 Example










sqr <- function(x){


x^2


}
 



This function is called sqr. 

It has one argument, called x.

Whatever value is inputted for x will be squared and the result outputted to the screen. 

This function must be loaded into R and can then be called.

We can call the function using:








sqr(x = 4)


#sqr(4)


[1] 16
 




To store the result into a variable x.sq








x.sq = sqr(x = 4)


 # x.sq = sqr(4)


> x.sq


[1] 16
 




More Complicated Example








sf2 <- function(a1, a2, a3){


x <- sqrt(a1^2 + a2^2 + a3^2)


return(x)


}

 
 





This function is called sf2 with 3 arguments. 

The values inputted for a1, a2, a3 will be squared, summed and the square root of the sum calculated and stored in x. 

(There will be no output to the screen as in the last example.)

The return command specifies what the function returns, here the value of x. 

We will not be able to view the result of the function unless we store it.









sf2(a1=2, a2=3, a3=4)


sf2(2, 3, 4)                                                 # Can't see result.


res = sf2(a1=2, a2=3, a3=4)


res = sf2(2, 3, 4)                                        # Need to use this.


res


[1] 5.385165
 




We can also give some/all arguments default values.









mypower <- function(x, pow=2){

x^pow

}
 





If a value for the argument pow is not specified in the function call, a value of 2 is used. If a value for “pow” is specified, that value is used.









mypower(4)

[1] 16

mypower(4, 3)

[1] 64

mypower(pow=5, x=2)                                                         

[1] 32

 
 






--------------------------------------------------------------------------------


More Complex Functions

The following function returns several values in the form of a list:


myfunc <- function(x)

{

# x is expected to be a numeric vector

# function returns the mean, sd, min, and max of the vector x

the.mean <- mean(x)

the.sd <- sd(x)

the.min <- min(x)

the.max <- max(x)

return(list(average=the.mean,stand.dev=the.sd,minimum=the.min,

maximum=the.max))

}


