
Numerical Computation (MS4024) 


In this module we will learn two very important programming languages used in numerical calculations.


R - Statistical Programming (Todays’s Class)


MATLAB - (“Mat” as in matrices) MATLAB, developed by MathWorks Inc, is a high-level language and interactive environment for numerical computation, visualization, and programming. Using MATLAB, you can analyze data, develop algorithms, and create models and applications. 


We will also be using LATEX to prepare reports.  In previous years, one teaching week was set aside for instruction on LATEX. This year, there will be short periods of instruction on an ongoing basis. 

Grading

The module will be assessed by a series of assessments during the period of Part 1 (weighted between 5% and 20%).  Assessments will comprise of written reports and in-class examinations. All project work must be submitted in documents prepared in LATEX.


Approximately 45% of the marks will be allocated to each of the two components, with another 10% for a project relevant to industrial applications of numerical computation, or similar matters.


There will be no end-of-semester assessment.


Teaching Week 13 (i.e. “Reading Week” ) will be a normal instructional week for this module. All project work must be submitted by the end of this week.


Instructional Material 

Instructional material will be distributed to the class using the SULIS system. Please advise me if you have any difficulties accessing this material as soon as possible.





The R environment


R is an integrated suite of software facilities for data manipulation, calculation and graphical display. It includes

•
an effective data handling and storage facility,

•
a suite of operators for calculations on arrays, in particular matrices,

•
a large, coherent, integrated collection of intermediate tools for data analysis,

•
graphical facilities for data analysis and display either on-screen or on hardcopy, and

•
a well-developed, simple and effective programming language which includes conditionals, loops, user-defined recursive functions and input and output facilities.



Syllabus

Language essentials: Objects; functions; vectors; missing values; matrices and arrays; factors; lists; data frames. 

Indexing, sorting, and implicit loops. Logical operators. Packages and libraries.  

Flow control: for, while, if/else, repeat, break. 

Probability distributions: Built-in distributions in R; densities, cumulatives, quantiles, random numbers. 

Statistical graphics: Graphical devices. High level plots. Low level graphics functions. 

Statistical functions: One- and two-sample inference, regression and correlation, tabular data, power, sample size calculations.


















The Matlab language


The syllabus for the MATLAB component of the course is:

•
Introduce MATLAB command syntax; MATLAB workspace, arithmetic, number formats, variables, built-in functions.

•
Using vectors in MATLAB ; colon notation.

•
Arrays; array indexing, array manipulation.

•
Two-dimensional graphics; basic plots, axes, multiple plots in a single figure, saving and printing figures.

•
Matlab commands in “batch" mode; script M-files, saving variables to a file, the diary function.

•
Relational and logical operations; testing for equality/inequality, and/or/not.

•
Control flow: for, while, if/else, case, try/catch.

•
Function M-files: parameter passing mechanisms, global and local variables.



Applications of Matlab; topics to be taken from:

•
Numerical Linear Algebra; norms and condition numbers, solution of linear equations, inverse, pseudo-inverse and determinant, LU and Cholesky factorisations, QR factorisation, Singular Value Decomposition, eigenvalue problems.

•
Polynomials and data fitting.

•
Nonlinear equations and optimisation.

•
Numerical solution of ordinary differential equations.










 Starting R


R can be started in the usual way by double-clicking on the R icon on the desktop.
•
R works best if you have a dedicated folder for each separate project - called the working folder.

•
Create the directory/folder that will be used as the working folder, e.g. create a folder on your desktop titled Your_name by right-clicking, then clicking New > Folder.

•
Right-click on an existing R icon and click Copy.

•
In the working folder, right-click and click Paste. The R icon will appear in the folder.



