


Module MS4024 Numerical Computation - R Component
Introduction to Statistical Computing
Introduction to R
What is R?
Using R
Packages
Defining Variables
Using the Help functions

\end{frame}
%=====================================================================================================================%
\begin{frame}

Introduction to R

R is the product of an active movement among statisticians for a powerful, programmable, portable, and open computing environment, applicable to the most complex and sophsticated problems, as well as“routine”analysis.
 
There are no restrictions on access or use.


Statisticians have implemented hundreds of specialised statistical procedures for a wide variety of applications as contributed packages, which are also freely-available and which integrate directly into R.

 
R has data handling and storage facilities, a suite of operators for calculations on arrays in particular matrices. A large coherent integrated collection of intermediate tools for data analysis
A large selection of demonstration datasets used in the illustration of many statistical methods.
Graphical facilities for data analysis and display either directly.at the computes or on hardcopy.
 
\end{frame}
%=====================================================================================================================%
\begin{frame}
R is a flexible language that is object oriented and thus allows the manipulation of a complex data structures in a condensed and efficient manner.


R's graphical abilities are also remarkable with possible interfacing with text processors such as Latex with the package sweave.

R offers the addtional advantage of being a free and opensource system under the GNU general public licence.

R is primarily a statistical language. R can be installed free of charge from www.r-project.org
\end{frame}
%=====================================================================================================================%
\begin{frame}
An online guide "An Introduction to R" can be access by typing help.start() at the command prompt to access this.

R is a statistical Environment for statistical computing and graphics, which is available for windows, Unix and Mac OS platforms.

R is maintained and distributed by an international team of statisticians and computers scientists.
R is one of the major tools used in statistical research and in applications of statistics research.

R is an open-source (GPL) statistical environment modeled after S and S-Plus. The S language was developed in the late 1980s at AT&T labs. The R project was started by Robert Gentleman and Ross Ihaka of the Statistics Department of the University of Auckland in 1995. It has quickly gained a widespread audience. 
\end{frame}
%=====================================================================================================================%
\begin{frame}
R is currently maintained by the R core-development team, a hard-working, international team of volunteer developers. 

The R project web page ( http://www.r-project.org ) is the main site for information on R. At this site are directions for obtaining the software, accompanying packages and other sources of documentation.

\end{frame}
%=====================================================================================================================%
\begin{frame}
\frametitle{Using R}
R is most easily used in an interactive manner, typing code into the command line and R gives you an response. Questions are asked and answered on the command line. To start up R's command line you can do the following: in Windows find the R icon and double click. Other operating systems may have different ways.

R can be started in the usual way by double-clicking on the R icon on the desktop.

\end{frame}
%=====================================================================================================================%
\begin{frame}
The > is called the prompt, used to indicate where you are to type. If a command is too long to fit on a line, a + is used for the continuation prompt.
If a command is not complete at the end of a line, R will give the "+" prompt on second and subsequent lines and continue to read input until the command is syntactically complete.
\end{frame}
%=====================================================================================================================%
\begin{frame}
Commands are separated either by a semi-colon (;), or by a newline. Elementary commands can be grouped together into one compound expression by braces ({ and }). 

Comments can be put almost anywhere, starting with a hashmark (#); everything after "#" is a comment.
\end{frame}
%=====================================================================================================================%
\begin{frame}
The R console opens with information and then a prompt mark  >  it is ready to accept commands.
\end{frame}
%=====================================================================================================================%
\begin{frame}
The Assignment operator

The assignment operator is a "=". This is valid as of R version 1.4.0. Previously it was (and still can be) a "<-".
Both will be used, although, you should learn one and stick with it.
\end{frame}
%=====================================================================================================================%
\begin{frame}
Defining Variables 
R is case sensitive.
A convention is to use define a variable name with a capital letter. This reduces the chance of overwriting inbuild R functions, which are usually written in lowercase letters.
\end{frame}
%=====================================================================================================================%
\begin{frame}
Commonly used variable names such as x,y and z (in lower case letters) are not "reserved".

x = 2           # create variable x and assign the value 2
y <- 4          # create variable y and assign the value 4
\end{frame}
%=====================================================================================================================%
\begin{frame}5 -> z          # create variable z and assign the value 5

x  #print x t\end{frame}o screen
y  #print y to screen
z  #print z to screen
\end{frame}
%=====================================================================================================================%
\begin{frame}
The value of each variable is prefaced with a " [1] ". This indicates that the value is a vector. More on that later.
%=====================================================================================================================%




Basic Calculations

We will brief\begin{frame}ly look at how R accomplished basic calculations.


x*y			# multiplication
x/z			# division

x^2			# powers
sqrt(x)		# square root

exp(z)		 # exponentials   
log(y)\end{frame}		 # logarithms

pi             # returns the value of pi to six decimal places

 
Complex numbers , Trigonometric  Functions and Binomial Coefficients
 
%=====================================================================================================================% 
Binomial coefficients are computed using the choose() command.


 

J = -1 ;  sqrt(J)  ;  str(J) ;      # variable is defined as numeric, not complex.
K = -1 +0i ;\begin{frame}  sqrt(K)  ;  str(K) ;  # variable is defined as complex .


sin(3.5*pi)             # correct answer is -1
cos(3.5*pi)             # correct answer is zero
 
choose(6,2)             # From 6 how many ways of choosing items.
 
Data Vectors

x=c(1,3,4,5,6,7)
y=c("R","G","B")

\end{frame}
%=====================================================================================================================%
\begin{frame}




Packages
"R"contains one or more libraries of packages. Packages contain various functions and data sets for numerous purposes, e.g.survival package, genetics package, fda package, etc. Some packages are part of the basic installation. 
R consists of a base package and many additional packages.

Others can be downloaded from the Comprehensive R Archive Network (CRAN).

To access all of the functions and data sets in a particular package, it must be loaded into the workspace. 
\end{frame}
%=====================================================================================================================%
\begin{frame}

For example, to load the fda package:

> library(fda)

One important thing to note is that if you terminate your session and start a new session with the saved workspace, you must load  the packages again



install.packages("evir")
 
To get out of R, just type: q(). 

\end{frame}
%=====================================================================================================================%
\begin{frame}

Using R's help commands

R has an inbuilt help facility. To get more information on any specific named function, for example “boxplot”, the command is: ?boxplot


?boxplot		# access help on boxplots
help(Im)        # access help on "Im"

\end{frame}
%=====================================================================================================================%
\begin{frame}

On most R installations help is available in HTML format by running help.start() which will launch a Web browser that allows the help pages to be browsed with hyperlinks. 
 



help.start()

 
 

R was originally designed as a command language.  
Commands were typed into a text-based input area on the computer screen and the program responded with a response to each command.

R is an open source software package, meaning that the code written to implement the various functions can be freely examined and modified.
R can be installed free of charge from the R-project website.

An online manual in “An introduction to R” is available via the R help system. Type “help.start ()” at the command prompt to access it, R   has many features in common with functional and object oriented programming languages.



In particular functions in R are treated as objects that can be manipulated or used recursively, example TSA book3

In common with functional languages, assignments in R can be avoided, but they are useful for clarity and convenience. In addition R runs faster when loops are avoided, which can often be achieved using matrix calculation instead however, thus results in obscure looking code.

\end{frame}
%=====================================================================================================================%
\begin{frame}

Functions in R can be treated as "objects" that can be manipulated or used recursively.

R shares many aspects with both Object orietated and functional programming languages. all data in R is stored an objects, which have a range of "methods" available. The "class" of an object can be found using the class() function.
Using the Help functions

\end{frame}
%=====================================================================================================================%
\begin{frame}
Embedded help commands "help()" and "help.search()" are good starting points to gather information.
Note that "help.search()" opens a web browser linked to the local manual pages.
 

R works best if you have a dedicated folder for each separateproject - called the working folder.Create the directory/folder that will be used as the working folder, e.g. create a folder on your desktop titled Your_name by right-clicking, then clicking New > Folder. 
 
Right-click on an existing R icon and click Copy.   In the working folder, right-click and click Paste. The R icon will appear in the folder. As with many advanced programming languages, R distinguishes between several types of object. Those types includes scalar, vector, matrix, time, series data frames functions and graphics. The R function str applied to any R onject, including R functions.To start R in Windows, double click the R icon. To start R in Unix or Linux, type ‘R’ at the command prompt. 

\end{frame}
%=====================================================================================================================%
\begin{frame}

Section 1: Basic R commands and Functions
Installing R on your computer
Editing your Data
Getting help in R
Manipulating Characters
Objects
Packages
Logical and Relational Operators
Generating Random Numbers
Section 2: Basic Mathematical operations
Trigonometric and power functions
Integration
Complex numbers
Matrices and Linear Algebra
Factorials and permutations
Section 3 : Data Structures
Vectors, Arrays and Matrices
Lists
Frames
Indexing
Subsetting
Section 4 : Regression models
Simple Linear Regression
Multiple Linear Regression
Non Linear Regression
Quadratic Regression
Histogram: sample code
Data Management
Vector Functions
Section 6: Simulation and Probability Distributions
Probability Distributions
Probability
Truncation and discretization
Creating New Functions
Argument Matching
Section 8: Hypothesis Tests
The Correlation Test
The Chi Square Test
The Chi Square Test for Independence
The PropTest
The F Test
The Kolmogorov Smirnov Test
The Anderson Darling Test
The t-test
The ANOVA F-Test
Section 9: Graphics
Graphics Parameters
Trellis Graphs
Customizing Trellis Graphics
Section 10: Simulation
Simulation Study : Random Walks
Simulation Study: Distribution of pairwise maxima and minima
Simulation Study : Gamblers Ruins
Simulation Study: Probability of Gambler Ruin

\end{frame}
%=====================================================================================================================%
\begin{frame}

Section 1: Basic R commands and Functions
Installing R on your computer
R can be easily downloaded from the Comprehenive R Archive Network (CRAN) website.
\end{frame}
%=====================================================================================================================%
\begin{frame}

Editing your Data

x=c(0 ,5)     	      # create a vector x
data.entry(x)  	   # edit the values using spreadsheet interface.
x  	                     # print to screen
x=edit(x)	          # the 'edit' function to call the script editor
x  	                     # print to screen

\end{frame}
%=====================================================================================================================%
\begin{frame}

Getting help in R
R has a built-in help facility. To get more information on any specific function, e.g. sqrt(), the command is
> help(sqrt)
An alternative is
> ? sqrt
We can also obtain help on features specified by special characters.
The argument must be enclosed in single or double quotes (e.g. "[[")
> 
Help is also available in HTML format by running
> help.start()


\end{frame}
%=====================================================================================================================%
\begin{frame}
help (sqrt)   	      # create a vector x
?exp  	   		  # edit the values using spreadsheet interface.
help("[[")	         # print to screen

\end{frame}
%=====================================================================================================================%
\begin{frame}
Manipulating Characters
> nchar("oscar")
[1] 5

Objects
During an R session, objects are created and stored by name. The command "ls()" displays all currently-stored objects (workspace). Objects can be removed using the "rm()" function.

\end{frame}
%=====================================================================================================================%
\begin{frame}

ls()
rm(x, a, temp, wt.males)
rm(list=ls())								#removes all of the objects in the workspace.


At the end of each R session, you are prompted to save your workspace. If you click Yes, all objects are written to the ".RData" file. 
When R is re-started, it reloads the workspace from this file and the command history stored in ".Rhistory" is also reloaded.
Packages
One of the strengths of R is the possibility to add functionalities to the base system by means of add-on packages. Simply put, a package is a coherent set of functions and
data.
 


Base packages
Base packages are considered part of the R source code. These packages contain the basic functions that allow R to work, and include many commonly used datasets and standard statistical and graphical functions. They should be automatically available in any R installation.


To see which packages are installed on your computer, use the library() command, with no arguments.

Contributed packages
Hundreds of contributed packages covering a wide array of modern statistical methods are available from the Comprehensive R Archive Network (CRAN; http: //cran.r-project.org).

To access all of the functions and data sets in a particular package, it must be loaded into the workspace. 
For example, to load the lattice package:
> library(lattice)

To remove a package from a session, use the detach command
>detach(package:lattice)

Packages must be reloaded at the start of each new session.


library(MASS)							  #loads the MASS package

install.packages("evir")			#install the "evir" package (select the Irish CRAN Mirror)
library(evir)

Logical and Relational Operators
Logical operators

 ==
Equality
 ! = 
 not equal to
 < 
Less Than
 >=
 greater than or equal to
>
Greater Than
 <=
 Less than or equal to
 


Generating Random Numbers
R is very useful for performing simulations.


runif(1)										 #generate a random number between 0 and 1
runif(4,min=0,max=6) 			  #generate four random numbers between 0 and 6

Random numbers can be discretized using the "floor()" or "Ceiling()" functions. Suppose we wish to simulate four throws of a dice.

X = ceiling (runif(4,min=0,max=6))
Y = floor (runif(4,min=1,max=7))
X+Y


Section 2: Basic Mathematical operations
Trigonometric and power functions

Integration

integrate(sin, lower =0, upper = 3)
integrate(dnorm, -1.96, 1.96)					 # standard normal distribution
integrate(dnorm, 0, Inf)							   # standard normal distribution

Complex numbers 

Matrices and Linear Algebra
Factorials and permutations
The Choose Function

63=654321= 20



> factorial(6)
[1] 720
> choose(6,3)
[1] 20
> 




Section 3 : Data Structures
Vectors, Arrays and Matrices

Lists

Frames

Indexing

Subsetting

Section 4 : Regression models

Simple Linear Regression


Summary of SLR


SLR analysis involves

Creating a scatterplot to determine the nature of the relationship between x and y

If the relationship is linear, measuring the strength of the relationship using the correlation coe cient

Fitting the best model by estimating parameter values from data

There are always lots of di erent possible models to describe a given data set

Using diagnostic plots of the residuals to check the adequacy of the fitted model. Must check for non-constant variance and non-normal errors.

If the relationship is non-linear, t e.g. polynomial, exponential, non-linear model and use predict to generate the fitted curve for plotting



lm (y ~ x)
Multiple Linear Regression
variable selection procedures

Backward Selection

Forward Selection

Stepwise Selection
Non Linear Regression
Quadratic Regression





Histogram: sample code
x <- rnorm(1000) 
hx <- hist(x, breaks=100, plot=FALSE) 
plot(hx, col=ifelse(abs(hx$breaks) < 1.669, 4, 2)) 

# What is cool is that "col" is supplied a vector.

Data Management

Creating New Variables
Operators
Built-in Functions
Control Structures
User-defined Functions
Sorting Data
Merging Data
Aggregating Data
Aggregating Data
Reshaping Data
Subsetting Data
Data Type Conversion
Vector Functions

range(x) 	    # range
sum(x) 		  # sum
min(x)	        # minimum
max(x)	       # maximum
diff(x, lag=2)   # lagged differences, with lag indicating which lag to use
Section 6: Simulation and Probability Distributions
Probability Distributions
  
Distribution
R name
Additional Arguments
beta
beta
shape1, shape2, ncp
binomial
binom
size, prob
Cauchy
cauchy
location, scale
Chi-squared
chisq
df, ncp
exponential
exp
rate
F
f
df1, df2
geometric
geom
prob
hypergeometric
hyper
values m,n,k
normal
norm
mean,sd
Poisson
pois
lambda
Student's
r
df,ncp
uniform
unif
min,max
Weibull
weibul
shape,scale
Wilcoxon
wilcox
m,n






Probability
 Simulation Study: Simulating dice throws

  


X = runif(100,0,6)

Truncation and discretization

The functions "floor" and "ceiling" can be used to discretize outcomes. In this instance we should use "ceiling".

ceiling(X)
floor(X)
round(X,2)

X=ceiling(X)

 

The expected value is 3.5.

The variance from first principles we can calculate the variance using
  





 

Creating New Functions
 
A simple function can be constructed as follows:

 
function_name <- function(arg1, arg2, ...){
commands
output
}

 
You decide on the name of the function.
The function command shows R that you are writing a function.
Inside the parenthesis you outline the input objects required and decide what to call them.
The commands occur inside the { }.
The name of whatever output you want goes at the end of the function.
Comments lines (usually a description of what the function does is placed at the beginning) are denoted by #.
 


sf1 <- function(x){
x^2
}


This function is called sf1. It has one argument, called x. Whatever value is inputted for x will be squared and the result outputted to the screen. This function must be loaded into R and can then be called.

 
Load the function into R by highlighting the code and clicking the Paste button in RWinEdt.
Type ls() into the console. Note that the function now appears. Can call the function using:
 


sf1(x = 3) sf1(3)
[1] 9 [1] 9



 
To store the result into a variable x.sq



x.sq <- sf1(x = 3) x.sq <- sf1(3)
> x.sq
[1] 9
 
sf2 <- function(a1, a2, a3){
x <- sqrt(a1^2 + a2^2 + a3^2)
return(x)
}



 
This function is called sf2 with 3 arguments. The values inputted for a1, a2, a3 will be squared, summed and the square root of the sum calculated and stored in x. (There will be no output to the screen as in the last example.) The return command speci es what the function returns, here the value of x. Will not be able to view the result of the function unless you store it.
 
sf2(a1=2, a2=3, a3=4) sf2(2, 3, 4) # Can't see result.
res <- sf2(a1=2, a2=3, a3=4) res <- sf2(2, 3, 4) # Need to use this.
res
[1] 5.385165
 
Argument Matching
 
How does R know which arguments are which? Uses argument matching.  Argument matching is done in a few different ways.
 
The arguments are matched by their positions. The rst supplied argument is matched to the rst formal argument and so on, e.g. when writing sf2 we speci ed that a1 comes rst, a2 second and a3 third. Using sf2(2, 3, 4) assigns 2 to a1, 3 to a2 and 4 to a3.
 
The arguments are matched by name. A named argument is matched to the formal argument with the same name, e.g.
sf2 arguments have names a1, a2 and a3. Can do things like sf2(a1=2, a3=3, a2=4), sf2(a3=2, a1=3, a2=4), etc.
 
Name matching happens first, then positional matching is used for any unmatched arguments.
 

Can also give some/all arguments default values.
 
mypower <- function(x, pow=2){
x^pow
}
 
If a value for the argument pow is not speci ed in the function call, a value of 2 is used.
 


mypower(4)
[1] 16




 
If a value for pow is specified, that value is used.
 



mypower(pow=5, x=2)
[1] 32
 




The following function returns several values in the form of a list:



myfunc <- function(x)
{
# x is expected to be a numeric vector
# function returns the mean, sd, min, and max of the vector x
the.mean <- mean(x)
the.sd <- sd(x)
the.min <- min(x)
the.max <- max(x)
return(list(average=the.mean,stand.dev=the.sd,minimum=the.min,
maximum=the.max))
}
 



x <- rnorm(40)
res <- myfunc(x)
res
$average
[1] 0.29713
$stand.dev
[1] 1.019685
$minimum
[1] -1.725289
$maximum
[1] 2.373015






 


To access any particular value use:
res$average
[1] 0.29713
res$stand.dev
[1]1.019685







 
Section 8: Hypothesis Tests

The Correlation Test

cor.test(X,Y)


The Chi Square Test





The Shapiro Wilk Test

The Shapiro Wilk Test is another test for distribution test





The Chi Square Test for Independence





The PropTest





The F Test





The Kolmogorov Smirnov Test

X = rnorm(5,1,5)
Y = rexp(5)
ks.test(X,Y)


The Anderson Darling Test






The t-test







The ANOVA F-Test

 
Section 9: Graphics


Graphics Parameters
Trellis Graphs
Trellis graphs are avialable throught the lattice package. A trellis graph displays a variable or the relationship between variables, conditioned on one or more other variables. Trellis graphs are available for a wide variety of plot types.
The typical format is
graph_type(formula, data=)
where graph_type is selected from the listed below. formula specifies the variable(s) to display and any conditioning variables . For example ~x|A means display numeric variable x for each level of factor A. y~x | A*B means display the relationship between numeric variables y and x separately for every combination of factor A and B levels.~x means display numeric variable x alone.
graph_type
description
formula examples
barchart
bar chart
x~A or A~x
bwplot
boxplot
x~A or A~x
cloud
3D scatterplot
z~x*y|A
contourplot
3D contour plot
z~x*y
densityplot
kernal density plot
~x|A*B
dotplot
dotplot
~x|A
histogram
histogram
~x
levelplot
3D level plot
z~y*x
parallel
parallel coordinates plot
dataframe
splom
scatterplot matrix
dataframe
stripplot
strip plots
A~x or x~A
xyplot
scatterplot
y~x|A
wireframe
3D wireframe graph
z~y*x
Here are some examples. They use the car data (mileage, weight, number of gears, number of cylinders, etc.) from the mtcars dataframe.
# Trellis Examples 
library(lattice) 
attach(mtcars)

# create factors with value labels 
gear.f<-factor(gear,levels=c(3,4,5),
   labels=c("3gears","4gears","5gears")) 
cyl.f <-factor(cyl,levels=c(4,6,8),
   labels=c("4cyl","6cyl","8cyl")) 

# kernel density plot 
densityplot(~mpg, 
   main="Density Plot", 
   xlab="Miles per Gallon")

# kernel density plots by factor level 
densityplot(~mpg|cyl.f, 
   main="Density Plot by Number of Cylinders",
   xlab="Miles per Gallon")

# kernel density plots by factor level (alternate layout) 
densityplot(~mpg|cyl.f, 
   main="Density Plot by Numer of Cylinders",
   xlab="Miles per Gallon", 
   layout=c(1,3))

# boxplots for each combination of two factors 
bwplot(cyl.f~mpg|gear.f,
   ylab="Cylinders", xlab="Miles per Gallon", 
   main="Mileage by Cylinders and Gears", 
   layout=(c(1,3))

# scatterplots for each combination of two factors 
xyplot(mpg~wt|cyl.f*gear.f, 
   main="Scatterplots by Cylinders and Gears", 
   ylab="Miles per Gallon", xlab="Car Weight")

# 3d scatterplot by factor level 
cloud(mpg~wt*qsec|cyl.f, 
   main="3D Scatterplot by Cylinders") 

# dotplot for each combination of two factors 
dotplot(cyl.f~mpg|gear.f, 
   main="Dotplot Plot by Number of Gears and Cylinders",
   xlab="Miles Per Gallon")

# scatterplot matrix 
splom(mtcars[c(1,3,4,5,6)], 
   main="MTCARS Data")

 
click to view
Note, as in graph 1, that you specifying a conditioning variable is optional. The difference between graphs 2 & 3 is the use of the layout option to contol the placement of panels.

Customizing Trellis Graphics


Unlike other R graphs, the trellis graphs described here are not effected by many of the options set in the par( ) function. To view the options that can be changed, look athelp(xyplot). It is frequently easiest to set these options within the high level plotting functions described above. Additionally, you can write functions that modify the rendering of panels. Here is an example.
# Customized Trellis Example
library(lattice)
panel.smoother <- function(x, y) {
  panel.xyplot(x, y) # show points 
  panel.loess(x, y)  # show smoothed line 
}
attach(mtcars)
hp <- cut(hp,3) # divide horse power into three bands 
xyplot(mpg~wt|hp, scales=list(cex=.8, col="red"),
   panel=panel.smoother,
   xlab="Weight", ylab="Miles per Gallon", 
   main="MGP vs Weight by Horse Power")




Section 10: Simulation

Simulation Study : Random Walks



P = 0.5 					#probability of a move to the right
Q = 1-P; S=Q/P;

Posn =0;N= 200;Trkr=numeric(N);Orgn=0; 

for (i in 1:N)
	{
	Trkr[i]=Posn
	if (P < runif(1)) Posn=Posn+1 else Posn=Posn-1
	if(Posn==0) Orgn=c(Orgn,i)
	}


plot(Trkr,type="o")
abline(h=0, col="red")
diff(Orgn)
Rogn = sort(diff(Orgn))
length(Rogn)
summary(as.factor(Rogn))
summary(as.factor(Rogn))[[1:10]]

Simulation Study: Distribution of pairwise maxima and minima



n=20
X<-rnorm(n) ; Y = rnorm(n); 
Mn =numeric(n) ;Mx = numeric(n);
for( i in 1:n)
{
W[i]=min(X[i],Y[i])
Z[i]=max(X[i],Y[i])
}

cbind(X,Y,W,Z)




Simulation Study : Gamblers Ruins
P = 0.55
A = 25
K = 40
N = 1000
M = 1000

Trk.A=numeric(N)
Trk.K=numeric(N)
Gamb.Win=0
Bank.Win=0
Dron=numeric(M)

for (j in 1:M){
for (i in 1:N)
{
Trk.A[i]=A;Trk.K[i]=K
if (P > runif(1)) {A=A+1;K=K-1;} else {A=A-1;K=K+1;}
if (A == -1) {break}
if (K == -1) {break}
Dron[j]=i
}
}


Trk.A= Trk.A[1:i]
Trk.K= Trk.K[1:i]

plot(Trk.A, ylim= c(-5,A+K+5),type="l",col="red")
points(Trk.K,,type="l",col="green")
abline(h=0 , lty =2 )
abline(h=0 ,lty=2)
abline(h=A+K, lty=2)
abline(v=0)



Simulation Study: Probability of Gambler Ruin
P = 0.51;  N = 2000; M = 5000; A.c = 15; K.c=15	#subscripts added to indicate constant values.

Gamb.Win=0;Bank.Win=0;Dron=numeric(M);

for (j in 1:M){
	A = A.c;K = K.c;			#reset to constant values for the start of each game
	for (i in 1:N)
	{
	if (P > runif(1)) {A=A+1;K=K-1;} else {A=A-1;K=K+1;}
	if (A == -1) {Bank.Win = Bank.Win +1; break; }
	if (K == -1) {Gamb.Win = Gamb.Win +1; break;}
	Dron[j]=i
	}

}
Dron
##



Simulation Study: estimating a quantile from a probability distributions



N = 2000 #number of Loops
n = 200 #size of each sample

Qvec = numeric(N)
Q = 0.975

for (i in 1:N){
X = rnorm(n)
Qtl = quantile (X,Q)
Qvec[i] = Qtl
}

Qvec

mean(Qvec)
# Alternative method

# Qtls[i] =quantile(rnorm(n),Q)

