
R for Finance


Actuarial packages

Risk Adjusted Performance Measures

Expected Shortfall



•
Time Series Analysis

•
Econometrics

•
Actuarial functions



 

Actuarial packages

Chain ladder

actuar package (Vincent Goulet)


portfolio analysis

Quantitate Risk Management (qrmlib)









install.packages(“PerformanceAnalytics”)

library(PerformanceAnalytics)


data(managers)

SterlingRatio(managers[,1:4])

CalmarRatio(managers[,1:4])
 






Risk Adjusted Performance Measures


The Calmar Ratio is used to determine return relative to drawdown (downside) risk in a hedge fund.


The Sterling Ratio is a ratio used mainly in the context of hedge funds. This risk-reward measure determines which hedge funds have the highest returns while enduring the least amount of volatility.









> CalmarRatio(managers$HAM4)

                  HAM4

Calmar Ratio 0.4227315

>

> SterlingRatio(managers$HAM2)

                                  HAM2

Sterling Ratio (Excess = 10%) 1.248598
 





Expected Shortfall


Expected shortfall (ES) is a risk measure, a concept used in finance (and more specifically in the field of financial risk measurement) to evaluate the market risk or credit risk of a portfolio. 


It is also called conditional VaR.


It is an alternative to value at riskthat is more sensitive to the shape of the loss distribution in the tail of the distribution. 

•
historical

•
modified

•
gaussian










> ES(edhec[,1:3], p=.95, method="historical")

   

    Convertible Arbitrage  CTA Global       Distressed Securities

ES  -0.09954768            -0.04284185      -0.06087217

> 

>

> ES(edhec[,1:3], p=.95, method="gaussian")


     Convertible Arbitrage  CTA Global      Distressed Securities

ES   -0.0348072             -0.04517756     -0.02976848

>

>

> ES(edhec[,1:3], p=.95, method="modified")


     Convertible Arbitrage  CTA Global      Distressed Securities

ES   -0.09954768            -0.04284185     -0.06087217
 


