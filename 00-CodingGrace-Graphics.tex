\chapter{ R Graphics}
\section{Enhancing your scatter plots}



%=====================================================================================================================%
\begin{frame}
\subsection{Adding a title }

It is good practice to label your scatterplots properly. You can specify the following argument
\begin{itemize}
	\item	main="Scatterplot Example", 	This provides the plot with a title
	\item	sub="Subtitle",                 This adds a subtitle
	\item	xlab="X variable ",				This command labels the x axis 
	\item   ylab="y variable ",				This command labels the y-axis
\end{itemize}
We can also add text to each margin, using the \texttt{mtext()} command.  
We simply require the number of the side. (1 = bottom, 2=left,3=top,4=right). 
We can change the colour using the col argument.
\footnotesize \begin{verbatim}
plot(x,y,main="Scatterplot Example",   sub="subtitle",    xlab="X variable ", ylab="y variable ")	
mtext("Enhanced Scatterplot", side=4,col="red ")
\end{verbatim}\normalsize
Alternatively , we can also use the command title() to add a title to an existing scatterplot.
\footnotesize \begin{verbatim}
title(main="Scatterplot Example)	
\end{verbatim}\normalsize


\section{Plot of single vectors}
If only one vector is specified i.e. \texttt{plot(x)},  the plot created will simply be a scatter-plot of the values of x against their indices.

$plot(x)$
Suppose we wish to examine a trend that these points represent. We can connect each covariate using a line.

$plot(x, type = "l")$
If we wish to have both lines and points, we would input the following code. This is quite useful if we wish to see how a trend develops over time.
$plot(x, type = "b")$














\section{Boxplot}
Boxplots can be used to identify outliers.

By default, the \texttt{boxplot()} command sets the orientation as vertical. By adding the argument \texttt{horizontal=TRUE}, the orientation can be changed to horizontal.
\footnotesize
\begin{framed}
	\begin{verbatim}
	boxplot(mtcars$mpg, horizontal=TRUE, xlab="Miles Per Gallon",
	main="Boxplot of MPG")
	\end{verbatim}
\end{framed}

%\begin{figure}
%  Requires \usepackage{graphicx}
%  \includegraphics[scale=0.4]{MTCARSboxplot.png}\\
%  \caption{Boxplot}\label{boxplot}
%\end{figure}



\newpage


\chapter{Data Visualization}
\section{Plots}
This section is an introduction for producing simple graphs with
the R Programming Language.
\begin{itemize}
	\item Line Charts  \item Bar Charts \item Histograms \item Pie
	Charts \item Dotcharts
\end{itemize}



\section{Plots}
This section is an introduction for producing simple graphs with
the R Programming Language.
\begin{itemize}
\item Line Charts  \item Bar Charts \item Histograms \item Pie
Charts \item Dotcharts
\end{itemize}


\begin{itemize}
\item
\item
\end{itemize}
\footnotesize \begin{verbatim}
> code here
 \end{verbatim}\normalsize


\subsection{ Charts}

\begin{myindentpar}{1cm}
\begin{verbatim}
# Define 2 vectors cars <- c(1, 3, 6, 4, 9) trucks <- c(2, 5, 4,
5, 12)

# Calculate range from 0 to max value of cars and trucks g_range
<- range(0, cars, trucks)

# Graph autos using y axis that ranges from 0 to max # value in
cars or trucks vector.  Turn off axes and # annotations (axis
labels) so we can specify them ourself plot(cars, type="o",
col="blue", ylim=g_range,
   axes=FALSE, ann=FALSE)

# Make x axis using Mon-Fri labels axis(1, at=1:5,
lab=c("Mon","Tue","Wed","Thu","Fri"))

# Make y axis with horizontal labels that display ticks at # every
4 marks. 4*0:g_range[2] is equivalent to c(0,4,8,12). axis(2,
las=1, at=4*0:g_range[2])

# Create box around plot box()

# Graph trucks with red dashed line and square points
lines(trucks, type="o", pch=22, lty=2, col="red")

# Create a title with a red, bold/italic font title(main="Autos",
col.main="red", font.main=4)

# Label the x and y axes with dark green text title(xlab="Days",
col.lab=rgb(0,0.5,0)) title(ylab="Total", col.lab=rgb(0,0.5,0))

# Create a legend at (1, g_range[2]) that is slightly smaller #
(cex) and uses the same line colors and points used by # the
actual plots legend(1, g_range[2], c("cars","trucks"), cex=0.8,
   col=c("blue","red"), pch=21:22, lty=1:2);

\end{verbatim}
\end{myindentpar}
\subsection{Bar charts}
\begin{myindentpar}{1cm}
\begin{verbatim}
# Define the cars vector with 7 values
cars <- c(1, 3, 6, 4, 9, 5, 7)
# Graph cars
barplot(cars)
\end{verbatim}
\end{myindentpar}
\subsection{Boxplots}
\subsection{Setting graphical parameters}
\subsection{Miscellaneous}
The following code can be used to make variations of the plots.

\begin{myindentpar}{1cm}
\footnotesize \begin{verbatim}
# Make an empty chart
plot(1, 1, xlim=c(1,5.5), ylim=c(0,7), type="n", ann=FALSE)

# Plot digits 0-4 with increasing size and color
text(1:5, rep(6,5), labels=c(0:4), cex=1:5, col=1:5)

# Plot symbols 0-4 with increasing size and color
points(1:5, rep(5,5), cex=1:5, col=1:5, pch=0:4)
text((1:5)+0.4, rep(5,5), cex=0.6, (0:4))

# Plot symbols 5-9 with labels
points(1:5, rep(4,5), cex=2, pch=(5:9))
text((1:5)+0.4, rep(4,5), cex=0.6, (5:9))

# Plot symbols 10-14 with labels
points(1:5, rep(3,5), cex=2, pch=(10:14))
text((1:5)+0.4, rep(3,5), cex=0.6, (10:14))

# Plot symbols 15-19 with labels
points(1:5, rep(2,5), cex=2, pch=(15:19))
text((1:5)+0.4, rep(2,5), cex=0.6, (15:19))

# Plot symbols 20-25 with labels
points((1:6)*0.8+0.2, rep(1,6), cex=2, pch=(20:25))
text((1:6)*0.8+0.5, rep(1,6), cex=0.6, (20:25))
\end{verbatim}\normalsize
\end{myindentpar}



