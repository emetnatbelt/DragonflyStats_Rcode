
\section{Discrete Probability Distributions}
\begin{itemize}
\item Poisson Distribution
\item Binomial Distribution
\item Geometric Distribution
\end{itemize}
\subsection{The Poisson distribution}
The Poisson distribution is characterized the by the arrival rate ‘lambda’.
\begin{framed}
	\begin{verbatim}
	rpois(n=5,lambda=4)          #generate five random numbers
	\end{verbatim}
\end{framed} 
\subsubsection{Simple population study}

Suppose a small island has population 1,000 at the start of a decade. The birth rate on this island is expected to 25 births per annum, while there is on average 10 deaths.  Forecast the population after five years.

\begin{framed}
	\begin{verbatim}
	Base = 1000
	Births =rpois(5,25)
	Deaths=rpois(5,10)
	Yrly.Incr =Births - Deaths
	Increase =cumsum(Yrly.Incr)
	Popn = Base + Births +Deaths
	\end{verbatim}
\end{framed} 
\subsection{The Binomial Distribution}
The binomial distribution is characterized by the number of trials, which in \texttt{R} is denoted as \texttt{‘size’} rather than ‘n’, and the probability of success  \texttt{‘prob’}.
\begin{framed}
	\begin{verbatim}
	rbinom(n=5,size=100,prob=0.25)           #generate five random numbers
	\end{verbatim}
\end{framed} 
\newpage
\section{Continuous Probability Distributions}


The continuous uniform distribution is commonly used in simulation.
\subsection{The Normal distribution}

The normal distribution is perhaps the most widely known distribution.
\begin{framed}
	\begin{verbatim}
	rnorm(n=15)       #15 random numbers, mean  = 0 , std. deviation = 1
	rnorm(n=15,mean= 17)       #set the mean to 17
	rnorm(n=15,mean= 17,sd=4)        #set the standard deviation to 4
	rnorm(15,17,4)                #argument matching : default positions
	\end{verbatim}
\end{framed} 






\subsection{The exponential distribution}


\end{document}
