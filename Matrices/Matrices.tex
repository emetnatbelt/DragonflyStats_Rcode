
 

Creating a matrix


Matrices can be created using the matrix() command. 


The arguments to be supplied are 


    1) vector of values to be entered

    2) dimensions of the matrix, specifying either the numbers of rows or columns.


Additionally you can specify if the values are to be allocated by row or column. 

By default they are allocated by column.









Vec1 = c(1,4,5,6,4,5,5,7,9)             # 9 elements


A = matrix(Vec1,nrow=3)                 #3 by 3 matrix. Values assigned by column.


A


C= matrix(  c(1,6,7,0.6,0.5,0.3,1,2,1), ncol=3 , byrow =TRUE)


C                                               #3 by 3 matrix. Values assigned by row.

 


Accessing Rows and Columns


Particular rows and columns can be accessed by specifying the row number or column number, leaving the other value blank.









A[1,]   # access first row of A


C[,2]   # access second column of C
 




Basic Matrix Calculations


1) Inverting a matrix 


To invert a matrix we use the command solve() with no additional argument.


We can use this same command to solve a system of linear equations Ax=b

We would specify the vector b as the additional argument.



2) Computing the determinant


To compute the determinant, the command is simply det()



3) Compute the transpose 


To compute the transpose of matrix A, we use the command t().



4) Determining the dimensions 


To find the dimensions of matrix A, we use the dim() command


5) Cross Products


We can compute cross products using the crossprod () command. 









solve(A)                            # Inverse of Matrix A    


t(A)                                  # tranpose of A


det(A)                               # determinant of A  


det(t(A))                           # determinant of the transpose of A


A %*% C                           # Matrix Multiplication


C * A                                #Casewise Multiplication
 




 




Linear Algebra Functions


R supports many import linear algebra functions such as cholesky decomposition, trace, rank, eigenvalues etc.


The required results may be determinable from the output of a command that pertains to an overall approach.









eigen(A)       #eigenvalues and eigenvectors       

                                                                      

qr(A)            #returns Rank of a matrix


svd(A)
 




 








Using rbind() and cbind()

Another methods of creating a matrix is to “bind” a number of vectors together, either by row or by column. 

The commands are rbind() and cbind() respectively.








> x1 =c(1,2) ; x2 = c(3,8)                                                


> D= rbind(x1,x2)


> E = cbind(x1,x2)


> det(D)


[1] 2


> det(E)


[1] 2

 







Part 2 Revision on Earlier Material
"	Accessing a column of a data frame
"	Accessing a row of a data frame

A particular row can be accessed by specifying the row index , while leaving the column index empty

Info [4,]			# Fourth row of "Info" is called

A sequence of rows can be accessed by specifying a sequence of rows as follows.

Info [10:15,]		# tenth row to fifteenth row of "Info" is called


\subsection{Subsetting datasets by rows}

Suppose we wish to divide a data frame into two different section. The simplest approach we can take is to create two new data sets, each assigned data from the relevant rows of the original data set.

Suppose our dataset ``Info" has the dimensions of 200 rows and 4 columns. We wish to separate "Info" into two subsets , with the first and second 100 rows respectively. ( We call these new subsets "Info.1" and "Info.2".)
\begin{verbatim}
Info.1 = Info[1:100,]		#assigning "info" rows 1 to 100
Info.2 = Info[101:200,]		#assigning "info" rows 101 to 200
\end{verbatim}

More useful commands such as rbind() and cbind()  can be used to manipulate vectors.

