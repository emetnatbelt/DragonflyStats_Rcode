

\chapter { R Graphics}
\section Enhancing your scatter plots








\section{Plot of single vectors}
If only one vector is specified i.e. \texttt{plot(x)},  the plot created will simply be a scatter-plot of the values of x against their indices.

$plot(x)$
Suppose we wish to examine a trend that these points represent. We can connect each covariate using a line.

$plot(x, type = "l")$
If we wish to have both lines and points, we would input the following code. This is quite useful if we wish to see how a trend develops over time.
$plot(x, type = "b")$











%----------------------------------------------------------------------------Graphical Methods--%
\newpage
\chapter{Graphical methods}

\section{Scatterplots}
\begin{figure}
  % Requires \usepackage{graphicx}
  \includegraphics[scale=0.40]{MTCARSmpgwt.png}\\
  \caption{Scatterplot}\label{mpgwt}
\end{figure}


\section{Adding titles, lines, points to plots}


\footnotesize \begin{verbatim}
library(MASS)
# Colour points and choose plotting symbols according to a levels of a factor
plot(Cars93$Weight, Cars93$EngineSize, col=as.numeric(Cars93$Type),
pch=as.numeric(Cars93$Type))

# Adds x and y axes labels and a title.
plot(Cars93$Weight, Cars93$EngineSize, ylab="Engine Size",
xlab="Weight", main="My plot")
# Add lines to the plot.
lines(x=c(min(Cars93$Weight), max(Cars93$Weight)), y=c(min(Cars93$EngineSize),
max(Cars93$EngineSize)), lwd=4, lty=3, col="green")
abline(h=3, lty=2)
abline(v=1999, lty=4)
# Add points to the plot.
\end{verbatim}\normalsize

\newpage

Histograms



Stem and Leaf Plots

\subsection{Bivariate Data}
\begin{verbatim}
Simple Scatterplots, Correlation and Covariance
X1 =
Y1 =
Plot(X1,Y1)
cor(X1)
cov(Y1)
\end{verbatim}
\end{document}
%-----------------------------------------------------------------------------------------------%
