

\chapter { R Graphics}
\section Enhancing your scatter plots
\subsection{Adding lines}
Previously we have used scatter plots to plot bivariate data. They were constructed using the plot() command.
Recall that we can use the arguments \texttt{xlim} and \texttt{ylim} to control the vertical and horizontal range of the plots, by specifying a two element vector (min and max) for each.

Using the \texttt{abline()} command, we can add lines to our scatter plots. We specify the argument according to the type of line required. A demonstration of three types of line is provided below.
Additionally we change the colour of the added lines, by specifying a colour in the \texttt{col} argument. We can also change the line type to one of four possible types, using the \texttt{lty} argument.

The line types are follows
\begin{itemize}
\item	\texttt{lty =1}   Normal full line (default)
\item	\texttt{lty =2}   Dashed line
\item	\texttt{lty =3}   Dotted line
\item	\texttt{lty =4}   Dash-dot line
\end{itemize}
\footnotesize \begin{verbatim}
x=rnorm(10)
y=rnorm(10)
plot(x,y)
plot(x,y,xlim=c(-4,4),ylim=c(-4,4))
abline(v =0 , lty =2 )    # add a vertical dotted line (here the y-axis) to the plot
abline(h=0  ,lty =3)    # add a horizontal dotted line (here the x-axis) to the plot
abline(a=0,b=1,col="green") # add a line to your plot with intercept "a" and slope "b"
 \end{verbatim}\normalsize

\subsection{Changing your plot character}

To change the plot character (the symbol for each covariate, we supply an additional argument to the plot() function.  This argument is formulated as pch=n where n is some number.
Additionally we change the colour of the characters, by specifying a colour in the col argument.
\footnotesize \begin{verbatim}
plot(x,y,pch=15,col="red")		#Square plot symbols
plot(x,y,pch=16,col="green")		#Orb plot symbols
plot(x,y,pch=17,col="mauve")		#Triangular plot symbols
plot(x,y,pch=36	,col="amber")		#Dollar sign plot symbols
\end{verbatim}\normalsize
Recall that we can add new variates to an existing scatterplot using the points() function. Remember to set the vertical and horizontal limits accordingly.
\footnotesize \begin{verbatim}
y1 = rnorm(10); y2 = rnorm(10)
plot(x,y1, pch=8,col="purple" ,xlim=c(-5,5),ylim=c(-5,5))
points(x,y2,pch=12,col="green")
\end{verbatim}\normalsize
\subsection{Adding the regression model line}

The \texttt{abline()} function can be used to add a regression model line  by supplying as an argument the \texttt{coef()} values for intercept and slope estimates .These estimates can be inputted directly by using both functions in conjunction.

\footnotesize \begin{verbatim}
Fit1 =lm(y1~x);  coef(Fit1)
abline(coef(Fit1))	
\end{verbatim}\normalsize

\subsection{Adding a title }

It is good practice to label your scatterplots properly. You can specify the following argument
\begin{itemize}
\item	main="Scatterplot Example", 	This provides the plot with a title
\item	sub="Subtitle",                 This adds a subtitle
\item	xlab="X variable ",				This command labels the x axis 
\item   ylab="y variable ",				This command labels the y-axis
\end{itemize}
We can also add text to each margin, using the \texttt{mtext()} command.  
We simply require the number of the side. (1 = bottom, 2=left,3=top,4=right). 
We can change the colour using the col argument.
\footnotesize \begin{verbatim}
plot(x,y,main="Scatterplot Example",   sub="subtitle",    xlab="X variable ", ylab="y variable ")	
mtext("Enhanced Scatterplot", side=4,col="red ")
\end{verbatim}\normalsize
Alternatively , we can also use the command title() to add a title to an existing scatterplot.
\footnotesize \begin{verbatim}
title(main="Scatterplot Example)	
\end{verbatim}\normalsize


\section{Combining plots}
It is possible to combine two plots. We used the graphical parameters command \texttt{par()} to create an array. 
Often we just require two plots side by side or above and below. We simply specify the numbers of rows and columns of this array using the \texttt{mfrow} argument, passed as a vector.

\begin{verbatim}
par(mfrow=c(1,2))
plot(x,y1)			# draw first plot
plot(x,y2)			# draw second plot
par(mfrow=c(1,1))		# reset to default setting.
\end{verbatim}

\section{Plot of single vectors}
If only one vector is specified i.e. \texttt{plot(x)},  the plot created will simply be a scatter-plot of the values of x against their indices.

$plot(x)$
Suppose we wish to examine a trend that these points represent. We can connect each covariate using a line.

$plot(x, type = "l")$
If we wish to have both lines and points, we would input the following code. This is quite useful if we wish to see how a trend develops over time.
$plot(x, type = "b")$











%----------------------------------------------------------------------------Graphical Methods--%
\newpage
\chapter{Graphical methods}

\section{Scatterplots}
\begin{figure}
  % Requires \usepackage{graphicx}
  \includegraphics[scale=0.40]{MTCARSmpgwt.png}\\
  \caption{Scatterplot}\label{mpgwt}
\end{figure}


\section{Adding titles, lines, points to plots}


\footnotesize \begin{verbatim}
library(MASS)
# Colour points and choose plotting symbols according to a levels of a factor
plot(Cars93$Weight, Cars93$EngineSize, col=as.numeric(Cars93$Type),
pch=as.numeric(Cars93$Type))

# Adds x and y axes labels and a title.
plot(Cars93$Weight, Cars93$EngineSize, ylab="Engine Size",
xlab="Weight", main="My plot")
# Add lines to the plot.
lines(x=c(min(Cars93$Weight), max(Cars93$Weight)), y=c(min(Cars93$EngineSize),
max(Cars93$EngineSize)), lwd=4, lty=3, col="green")
abline(h=3, lty=2)
abline(v=1999, lty=4)
# Add points to the plot.
\end{verbatim}\normalsize

\newpage

Histograms
Boxplots

%-----------------------------------------------------------------------------------------%
A box plot provides an excellent visual summary of many important aspects of a distribution. 
The box stretches from the lower hinge (defined as the 25th percentile) to the upper hinge (the 75th percentile) and therefore contains the middle half of the scores in the distribution.
A boxplot, or box and whisker diagram, provides a simple graphical summary of a set of data. It shows a measure of central location (the median), two measures of dispersion (the range and inter-quartile range), the skewness (from the orientation of the median relative to the quartiles) and potential outliers (marked individually). 
Boxplots are especially useful when comparing two or more sets of data. 
\begin{verbatim}
# boxplot on a formula:
boxplot.stats(count ~ spray, data = InsectSprays, col = "lightgray")
# *add* notches (somewhat funny here):
boxplot(count ~ spray, data = InsectSprays,        notch = TRUE, add = TRUE, col = "blue")
\end{verbatim}
%-------------------------------------------------------------------------------------------------%
Stem and Leaf Plots

\subsection{Bivariate Data}
\begin{verbatim}
Simple Scatterplots, Correlation and Covariance
X1 =
Y1 =
Plot(X1,Y1)
cor(X1)
cov(Y1)
\end{verbatim}
\end{document}
%-----------------------------------------------------------------------------------------------%
