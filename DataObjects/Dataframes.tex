


R Workshop

Introduction to R

Precision

Data Manipulation

data manipulation

Descriptive and Quantile Statistics

 

Data Frames

A data frame

•
 can be thought of as a data matrix or data set;

•
 is a generalization of a matrix;

•
 is a list of vectors and/or factors of the same length;

•
 has a unique set of row names



Data in the same position across columns come from the same experimental unit.


We can create data frames from pre-existing variables:


> d <- data.frame(mean_weight, Gender)




--------------------------------------------------------------------------------


Data set : mtcars


R contains several embedded data sets. One that is commonly used is mtcars


>mtcars


(for some background on mtcars, type ?mtcars)


Each of the columns has a name. Use the commands names() and dim() to find more information about mtcars.


Another useful command is summary().  It will give revelant information on the data object.



We can use the command rownames() to access the names of each row.


To access a particular column, use the dollar sign to specify it


>mtcars$mpg


We can now perform calculations on this column.






--------------------------------------------------------------------------------



