
%=====================================================================================================================%
\begin{frame}[fragile]
\frametitle{Graphical Data Entry Interface}
\begin{itemize}

\item The data.entry() command calls a spreadsheet graphical user
interface, which can be used to edit data. All changes are saved
automatically.




\item Alternatively, the edit() command calls the `R editor',
which can be used to edit specified data or the code used to
define that data.
\end{itemize}
\end{frame}
%=====================================================================================================================%
\begin{frame}[fragile]
\frametitle{Graphical Data Entry Interface}
\begin{itemize}

\begin{verbatim}
x<-edit(x)
\end{verbatim}

\end{itemize}
\end{frame}
%=====================================================================================================================%
\begin{frame}[fragile]
\frametitle{Graphical Data Entry Interface}
\begin{itemize}


In the last class, we looked at how to compute the mean, variance and standard deviation. 
 
As these are key outcomes of this part of the course, we shall briefly go over this material again 

\end{frame}
%=====================================================================================================================%
\begin{frame}[fragile]
\frametitle{Graphical Data Entry Interface}
\begin{itemize}

The mean
 
The mean (i.e. average) value is denoted with a bar over the set name i.e. " ".



     (pronounced “x bar”)  is the sample mean.

\end{frame}
%=====================================================================================================================%
\end{document}


x=c( 15,  34,  7,  12,  18,  9, 1,  42,  56,  28,  13,  24, 35)
boxplot(x, horizontal =TRUE)
