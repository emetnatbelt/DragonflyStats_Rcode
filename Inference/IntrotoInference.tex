\section{Introduction to Inference}
\begin{itemize}
\item Suppose we perform this experiment with an unknown die. The resultant sum of 100 throws is 301, (i.e. the resultant mean is 3.01). 
\item It is an unusual, but not impossible outcome from a fair die.
\item Consider another possibility: that the die is crooked, favouring low values. If this was the case, a sum of 301 would not be unusual.
\item 
Before we began the experiment we had no reason to suspect the die wasn't fair, and we expected a value of around 350. This is our null hypothesis.
\end{itemize}
\begin{description}
\item[H0:] The die is fair.
\item[HA:] The die is crooked.
\end{description}
%---------------------------------------------------------------------------------------------%
\subsection*{Understanding Standard Error}
\begin{itemize}
\item Let us repeat the same experiment as before, varying N the number of throws in each experiment.
\item Write down the standard deviation and variance for y for each of the following cases; N = 50,100,200,500
\item Recall from your previous statistics modules the concept of "Standard Error". This is the standard deviation of the sampling distribution. A key component in standard error is the sample size (analogous to the N value).
\end{itemize}
%---------------------------------------------------------------------------------------------%
\subsection*{P values}

The probability of getting a values as extreme or more as some statistic, such as sum or mean, is known as a p-value. When performing statistical calculations using computer software they are the most commonly used item for making statistical decisions.



In this last instance, we would usually fail to reject the null hypothesis. Many R outputs will give a group of asterisks beside the data to help the user in interpreting the data, depending on how significant the result is.

\begin{verbatim}
p-value  < 0.0001  	***
p-value  < 0.001	**
p-value  < 0.01	*
p-value  < 0.1
\end{verbatim}

For this module, as a rule of thumb, we will use the threshold of 0.01 for deciding whether to accept or reject the null hypothesis. If the p-value is less than 0.01 we reject the null hypothesis. If not, we fail to reject the null hypothesis.

\end{document}
