\newpage

\section{Hypothesis Tests for a Proportion}

We can also perform a hypothesis test for a population proportion “p”. 

The \texttt{R} function to carry out such a test is \texttt{prop.test}.

A manufacturer claims that the proportion of defective items produced is approximately 4\%. 
A random sample of size 50 is taken, 3 of which are defective. Is the manufacturer's claim justified?

The inputs to prop.test are 
\begin{description}
\item[x] - the number of defective items in the sample, 
\item[n] - the sample size, 
\item[p] - the probability being tested in the hypothesis	
\end{description}

(We can also specify the confidence  level and number of tails. See \texttt{?prop.test})
 
\begin{framed}
\begin{verbatim}
> prop.test(x=3, n=50, p = 0.04)

        1-sample proportions test with continuity correction

data:  3 out of 50, null probability 0.04 
X-squared = 0.1302, df = 1, p-value = 0.7182
alternative hypothesis: true p is not equal to 0.04 
95 percent confidence interval:
 0.01562459 0.17541874 
sample estimates:
   p 
0.06 
\end{verbatim}
\end{framed}
%=============================================================================%
The outputs are essentially the same as those for t.test:
- test statistic value,
- p-value,
- 95% CI,
- sample estimate, etc

%=============================================================================%
2 sample case

To compare the sample proportions of two groups, we require two inputs for the prop.test() command.

\begin{itemize}
\item[x] - the number of “successes” in the both samples ,specified as a vector of length 2.
\item[n] - the number of “trials” in the both samples, again specified as a vector of length 2.
\end{itemize}

Here the null hypothesis that both groups have the same proportion of “successes”, while the alternative is that they have different proportions of “successes”.
\begin{framed}
\begin{verbatim}
> prop.test(c(110,210),c(200,300))

        2-sample test for equality of proportions with continuity correction

data:  c(110, 210) out of c(200, 300) 
X-squared = 11.0768, df = 1, p-value = 0.0008742
alternative hypothesis: two.sided 
95 percent confidence interval:
 -0.24043848 -0.05956152 
sample estimates:
prop 1 prop 2 
  0.55   0.70
\end{verbatim}
\end{Framed}
%----------------------------------------------------------------------------------------%
\section{Exercise 4}
A poll on social issues interviewed 1025 people randomly selected from the United States. 450 of people said that they do not get enough time to themselves. A report claims that over 41\% of the population are not satisfied with personal time. Is this the case?
\begin{framed}
\begin{verbatim}

> prop.test(450,1025,p=0.40,alternative="greater")

        1-sample proportions test with continuity correction

data:  450 out of 1025, null probability 0.4
X-squared = 6.3425, df = 1, p-value = 0.005894
alternative hypothesis: true p is greater than 0.4
95 percent confidence interval:
 0.413238 1.000000
sample estimates:
        p
0.4390244
\end{verbatim}
\end{framed}
\subsection{Exercise 23b:}

A company wants to investigate the proportion of males and females promoted in the last year. 

45 out of 400 female candidates were promoted, while 520 out of 3270 male candidates were promoted. Is there evidence of sexism in the company?

\begin{verbatim}
> x.vec=c(45,520)
> n.vec=c(400,3270)
>  prop.test(x.vec,n.vec)
\end{verbatim}
\begin{framed}
\begin{verbatim}
 2-sample test for equality of proportions with continuity correction

data:  x.vec out of n.vec
X-squared = 5.5702, df = 1, p-value = 0.01827
alternative hypothesis: two.sided
95 percent confidence interval:
 -0.08133043 -0.01171238
sample estimates:
   prop 1    prop 2
0.1125000 0.1590214
\end{verbatim}
\end{framed}

\end{document}
