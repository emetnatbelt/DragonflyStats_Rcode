\section{Paired t-test}

In paired t-tests, we examine the case wise differences in both data sets. 
This procedure is of interest in studies with a “before / After” structure.  Both data sets must be of equal length. 

\begin{quoatation} 
Ho :  The average case-wise difference is zero
Ha :  The average case-wise difference is not zero.
\end{quotation}

\begin{verbatim}
# paired t-test
# t.test(y1,y2,paired=TRUE)  where y1 & y2 are numeric and are of the same length.
t.test(mod, unmod,paired=T)
\end{verbatim}
 
Another way of implementing the paired t-test is to first compute the vector of casewise differences, and then apply a simple one sample test.
\begin{framed}
\begin{verbatim}
# t.test(y1-y2) # Ho: mu=0
t.test(mod- unmod)
\end{verbatim}
 \end{framed}

\end{document}
